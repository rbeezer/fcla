Now we know that invertibility is equivalent to nonsingularity, and that the procedure outlined in \acronymref{theorem}{CINM} will always yield an inverse for a nonsingular matrix.  But rather than using that procedure, Sage implements a \verb?.inverse()? method.  In the following, we compute the inverse of a $3\times 3$ matrix, and then purposely convert it to a singular matrix by replacing the last column by a linear combination of the first two.
%
\begin{sageexample}
sage: A = matrix(QQ,[[ 3,  7, -6],
...                  [ 2,  5, -5],
...                  [-3, -8,  8]])
sage: A.is_singular()
False
sage: Ainv = A.inverse(); Ainv
[ 0  8  5]
[ 1 -6 -3]
[ 1 -3 -1]
sage: A*Ainv
[1 0 0]
[0 1 0]
[0 0 1]
sage: col_0 = A.column(0)
sage: col_1 = A.column(1)
sage: C = column_matrix([col_0, col_1, 2*col_0 - 4*col_1])
sage: C.is_singular()
True
sage: C.inverse()
Traceback (most recent call last):
...
ZeroDivisionError: input matrix must be nonsingular
\end{sageexample}
%
Notice how the failure to invert \verb?C? is explained by the matrix being singular.\par
%
Systems with nonsingular coefficient matrices can be solved easily with the matrix inverse.  We will recycle \verb?A? as a coefficient matrix, so be sure to execute the code above.
%
\begin{sageexample}
sage: const = vector(QQ, [2, -1, 4])
sage: A.solve_right(const)
(12, -4, 1)
sage: A.inverse()*const
(12, -4, 1)
sage: A.solve_right(const) == A.inverse()*const
True
\end{sageexample}
%
If you find it more convenient, you can use the same notation as the text for a matrix inverse.
%
\begin{sageexample}
sage: A^-1
[ 0  8  5]
[ 1 -6 -3]
[ 1 -3 -1]
\end{sageexample}
%
\begin{sageverbatim}
\end{sageverbatim}
%

