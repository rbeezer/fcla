%%%%(c)
%%%%(c)  This file is a portion of the source for the textbook
%%%%(c)
%%%%(c)    A First Course in Linear Algebra
%%%%(c)    Copyright 2004 by Robert A. Beezer
%%%%(c)
%%%%(c)  See the file COPYING.txt for copying conditions
%%%%(c)
%%%%(c)
%%%%%%%%%%%
%%
%%  Annotated Acronyms LT
%%  Linear Transformations
%%
%%%%%%%%%%%
%
\annoacro{theorem}{MBLT}{%
You give me an $m\times n$ matrix and I'll give you a linear transformation $\ltdefn{T}{\complex{n}}{\complex{m}}$.  This is our first hint that there is some relationship between linear transformations and matrices.
}
%
\annoacro{theorem}{MLTCV}{%
You give me a linear transformation $\ltdefn{T}{\complex{n}}{\complex{m}}$ and I'll give you an $m\times n$ matrix.  This is our second hint that there is some relationship between linear transformations and matrices.  Generalizing this relationship to arbitrary vector spaces (i.e.\ not just $\complex{n}$ and  $\complex{m}$) will be the most important idea of \acronymref{chapter}{R}.
}
%
\annoacro{theorem}{LTLC}{%
A simple idea, and as described in \acronymref{exercise}{LT.T20}, equivalent to the \acronymref{definition}{LT}.  The statement is really just for convenience, as we'll quote this one often.
}
%
\annoacro{theorem}{LTDB}{%
Another simple idea, but a powerful one.  ``It is enough to know what a linear transformation does to a basis.''  At the outset of \acronymref{chapter}{R}, \acronymref{theorem}{VRRB} will help us define a very important function, and then \acronymref{theorem}{LTDB} will allow us to understand that this function is also a linear transformation.
}
%
\annoacro{theorem}{KPI}{%
The pre-image will be an important construction in this chapter, and this is one of the most important descriptions of the pre-image.  It should remind you very much of \acronymref{theorem}{PSPHS}.  Also see \acronymref{theorem}{RPI}, which has a description below.
}
%
\annoacro{theorem}{KILT}{%
Kernels and injective linear transformations are intimately related.  This result is the connection.  Compare with \acronymref{theorem}{RSLT} below.
}
%
\annoacro{theorem}{ILTB}{%
Injective linear transformations and linear independence are intimately related.  This result is the connection.  Compare with \acronymref{theorem}{SLTB} below.
}
%
\annoacro{theorem}{RSLT}{%
Ranges and surjective linear transformations are intimately related.  This result is the connection.  Compare with \acronymref{theorem}{KILT} above.  
}
%
\annoacro{theorem}{SSRLT}{%
This theorem provides the most direct way of forming the range of a linear transformation.  The resulting spanning set might well be linearly dependent, and beg for some clean-up, but that doesn't stop us from having very quickly formed a reasonable description of the range.  If you find the determination of spanning sets or ranges difficult, this is one worth remembering.  You can view this as the analogue of forming a column space by a direct application of \acronymref{definition}{CSM}.
}
%
\annoacro{theorem}{SLTB}{%
Surjective linear transformations and spanning sets are intimately related.  This result is the connection.  Compare with \acronymref{theorem}{ILTB} above.
}
%
\annoacro{theorem}{RPI}{%
This is the analogue of \acronymref{theorem}{KPI}.  Membership in the range is equivalent to nonempty pre-images.
}
%
\annoacro{theorem}{ILTIS}{%
Injectivity and surjectivity are independent concepts.  You can have one without the other.  But when you have both, you get invertibility, a linear transformation that can be run ``backwards.''  This result might explain the entire structure of the four sections in this chapter.
}
%
\annoacro{theorem}{RPNDD}{%
This is the promised generalization of \acronymref{theorem}{RPNC} about matrices.  So the number of columns of a matrix is the analogue of the dimension of the domain.  This will become even more precise in \acronymref{chapter}{R}.  For now, this can be a powerful result for determining dimensions of kernels and ranges, and consequently, the injectivity or surjectivity of linear transformations.  Never underestimate a theorem that counts something.
}
%
% End LT.tex annotated acronyms
