%%%%(c)
%%%%(c)  This file is a portion of the source for the textbook
%%%%(c)
%%%%(c)    A First Course in Linear Algebra
%%%%(c)    Copyright 2004 by Robert A. Beezer
%%%%(c)
%%%%(c)  See the file COPYING.txt for copying conditions
%%%%(c)
%%%%(c)
\exercise{C05}{\robertbeezer} %  row-reduce all archetypes
%
\begin{exercisegroup}
\begin{para}For problems C10--C19, find all solutions to the system of linear equations.  Use your favorite computing device to row-reduce the augmented matrices for the systems, and write the solutions as a set, using correct set notation.\end{para}
%
\exercise{C10}{\robertbeezer} %  4x4 system, unique solution
\exercise{C11}{\robertbeezer} %  3x4 system, inconsistent
\exercise{C12}{\robertbeezer} %  3x4 system, 2 free variables
\exercise{C13}{\robertbeezer} %  3x4 system, inconsistent
\exercise{C14}{\robertbeezer} %  3x4 system, 2 free variables
\exercise{C15}{\robertbeezer} %  3x4 system, 1 free variable
\exercise{C16}{\robertbeezer} %  3x4 system, inconsistent
\exercise{C17}{\robertbeezer} %  4x4 system, unique
\exercise{C18}{\robertbeezer} %  3x5 system, 2 free variables
\exercise{C19}{\robertbeezer} %  4x2 system, unique
\end{exercisegroup}
%
%
\begin{exercisegroup}
\begin{para}For problems C30--C33, row-reduce the matrix without the aid of a calculator, indicating the row operations you are using at each step using the notation of \acronymref{definition}{RO}.\end{para}
%
\exercise{C30}{\robertbeezer}  % Row-reduce 3x4 by hand
\exercise{C31}{\robertbeezer}  % Row-reduce 3x3 by hand
\exercise{C32}{\robertbeezer}  % Row-reduce 3x3 by hand
\exercise{C33}{\robertbeezer}  % Row-reduce 3x4 by hand
\end{exercisegroup}
%
%
\exercise{M40}{\robertbeezer}  % Row space computations on row-equivalent 3x4 matrices
\exercise{M45}{\chrisblack}    % Lizards, mice, peacocks (oh my!)
\exercise{M50}{\robertbeezer}  % Cars, trucks, etc in parking lot
%
%
\exercise{T10}{\robertbeezer}   % Row operations are reversible
\exercise{T11}{\robertbeezer}   % Row-equivalence is an equivalence relation
\exercise{T12}{\robertbeezer}   % Contiguous rows, front columns still in rref
\exercise{T13}{\robertbeezer}   % Relax T12, rows not contiguous
