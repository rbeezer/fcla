%%%%(c)
%%%%(c)  This file is a portion of the source for the textbook
%%%%(c)
%%%%(c)    A First Course in Linear Algebra
%%%%(c)    Copyright 2004 by Robert A. Beezer
%%%%(c)
%%%%(c)  See the file COPYING.txt for copying conditions
%%%%(c)
%%%%(c)
For both parts, we need the extended echelon form of the matrix.
%
\begin{align*}
\begin{bmatrix}
 -7 & -11 & -19 & -15 & 1 & 0 & 0 & 0\\
 6 & 10 & 18 & 14 & 0 & 1 & 0 & 0 \\
 3 & 5 & 9 & 7 & 0 & 0 & 1 & 0 \\
 -1 & -2 & -4 & -3 & 0 & 0 & 0 & 1
\end{bmatrix}
\rref
\begin{bmatrix}
 \leading{1} & 0 & -2 & -1 & 0 & 0 & 2 & 5 \\
 0 & \leading{1} & 3 & 2 & 0 & 0 & -1 & -3 \\
 0 & 0 & 0 & 0 & \leading{1} & 0 & 3 & 2 \\
 0 & 0 & 0 & 0 & 0 & \leading{1} & -2 & 0
\end{bmatrix}
%
\end{align*}
%
From this matrix we extract the last two rows, in the last four columns to form the matrix $L$,
%
\begin{align*}
L
=
\begin{bmatrix}
\leading{1} & 0 & 3 & 2 \\
 0 & \leading{1} & -2 & 0
\end{bmatrix}
\end{align*}
%
(a)\quad By \acronymref{theorem}{FS} and \acronymref{theorem}{BNS} we have
%
\begin{align*}
\csp{D}=\nsp{L}=\spn{\set{
\colvector{-3\\2\\1\\0},\,
\colvector{-2\\0\\0\\1}
}}
\end{align*}
%
(b)\quad  By \acronymref{theorem}{FS} and \acronymref{theorem}{BRS} we have
%
\begin{align*}
\lns{D}=\rsp{L}=\spn{\set{
\colvector{1\\0\\3\\2},\,
\colvector{0\\1\\-2\\0}
}}
\end{align*}
%
