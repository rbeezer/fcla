%%%%(c)
%%%%(c)  This file is a portion of the source for the textbook
%%%%(c)
%%%%(c)    A First Course in Linear Algebra
%%%%(c)    Copyright 2004 by Robert A. Beezer
%%%%(c)
%%%%(c)  See the file COPYING.txt for copying conditions
%%%%(c)
%%%%(c)
Determine the kernel of $S$ first.  The condition that $\lt{S}{\begin{bmatrix}a & b\end{bmatrix}}=\zerovector$ becomes $(3a+b)+(5a+2b)x=0+0x$.  Equating coefficients of these polynomials yields the system
%
\begin{align*}
3a+b&=0\\
5a+2b&=0
\end{align*}
%
This homogeneous system has a nonsingular coefficient matrix, so the only solution is $a=0$, $b=0$ and thus
%
\begin{equation*}
\krn{S}=\set{\begin{bmatrix}0 & 0\end{bmatrix}}
\end{equation*}
%
By \acronymref{theorem}{KILT}, we know $S$ is injective.  With $\nullity{S}=0$ we employ \acronymref{theorem}{RPNDD} to find
%
\begin{equation*}
\rank{S}=\rank{S}+0=\rank{S}+\nullity{S}=\dimension{M_{12}}=2=\dimension{P_1}
\end{equation*}
%
Since $\rng{S}\subseteq P_1$ and $\dimension{\rng{S}}=\dimension{P_1}$, we can apply \acronymref{theorem}{EDYES} to obtain the set equality $\rng{S}=P_1$ and therefore $S$ is surjective. \par
%
One of the two defining conditions of an invertible linear transformation is (\acronymref{definition}{IVLT})
%
\begin{align*}
\lt{\left(\compose{S}{R}\right)}{a+bx}&=\lt{S}{\lt{R}{a+bx}}\\
&=\lt{S}{\begin{bmatrix}(2a-b) & (-5a+3b)\end{bmatrix}}\\
&=\left(3(2a-b)+(-5a+3b)\right)+\left(5(2a-b)+2(-5a+3b)\right)x\\
&=\left((6a-3b)+(-5a+3b)\right)+\left((10a-5b)+(-10a+6b)\right)x\\
&=a+bx\\
&=\lt{I_{P_1}}{a+bx}
\end{align*}
%
That 
$\lt{\left(\compose{R}{S}\right)}{\begin{bmatrix}a & b\end{bmatrix}}=
\lt{I_{M_{12}}}{\begin{bmatrix}a & b\end{bmatrix}}$
is similar.
