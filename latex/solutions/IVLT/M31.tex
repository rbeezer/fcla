%%%%(c)
%%%%(c)  This file is a portion of the source for the textbook
%%%%(c)
%%%%(c)    A First Course in Linear Algebra
%%%%(c)    Copyright 2004 by Robert A. Beezer
%%%%(c)
%%%%(c)  See the file COPYING.txt for copying conditions
%%%%(c)
%%%%(c)
(Another approach to this solution would follow \acronymref{example}{CIVLT}.)\par
%
We are given that $R$ is invertible.  The inverse linear transformation can be formulated by considering the pre-image of a generic element of the codomain.  With injectivity and surjectivity, we know that the pre-image of any element will be a set of size one --- it is this lone element that will be the output of the inverse linear transformation.\par
%
Suppose that we set $\vect{v}=\begin{bmatrix}x\\y\end{bmatrix}$ as a generic element of the codomain, $M_{21}$.  Then if $\begin{bmatrix}r & s\end{bmatrix}=\vect{w}\in\preimage{R}{\vect{v}}$,
%
\begin{align*}
\begin{bmatrix}x\\y\end{bmatrix}&=\vect{v}=\lt{R}{\vect{w}}\\
&=
\begin{bmatrix}
r+3s\\
4r+11s
\end{bmatrix}
\end{align*}
%
So we obtain the system of two equations in the two variables $r$ and $s$,
%
\begin{align*}
r+3s&=x\\
4r+11s&=y
\end{align*}
%
With a nonsingular coefficient matrix, we can solve the system using the inverse of the coefficient matrix,
%
\begin{align*}
r&=-11x+3y\\
s&=4x-y
\end{align*}
%
So we define,
%
\begin{equation*}
\lt{\ltinverse{R}}{\vect{v}}
=\lt{\ltinverse{R}}{\begin{bmatrix}x\\y\end{bmatrix}}
=\vect{w}
=\begin{bmatrix}r & s\end{bmatrix}
=\begin{bmatrix}-11x+3y & 4x-y\end{bmatrix}
\end{equation*}
%