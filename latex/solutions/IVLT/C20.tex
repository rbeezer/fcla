%%%%(c)
%%%%(c)  This file is a portion of the source for the textbook
%%%%(c)
%%%%(c)    A First Course in Linear Algebra
%%%%(c)    Copyright 2004 by Robert A. Beezer
%%%%(c)
%%%%(c)  See the file COPYING.txt for copying conditions
%%%%(c)
%%%%(c)
(a)  We will compute the kernel of $T$.  Suppose that $a+bx+cx^2\in\krn{T}$.  Then
\begin{equation*}
\begin{bmatrix}0 & 0 \\ 0 & 0\end{bmatrix}
=\lt{T}{a+bx+cx^2}
=
\begin{bmatrix}
a+2b-2c & 2a+2b \\
-a+b-4c & 3a+2b+2c
\end{bmatrix}
\end{equation*}
%
and matrix equality (\acronymref{theorem}{ME}) yields the homogeneous system of four equations in three variables,
\begin{align*}
a+2b-2c&=0\\ 
2a+2b&=0\\
-a+b-4c&=0\\
3a+2b+2c&=0\\
\end{align*}
%
The coefficient matrix of this system row-reduces as
\begin{equation*}
\begin{bmatrix}
 1 & 2 & -2 \\
 2 & 2 & 0 \\
 -1 & 1 & -4 \\
 3 & 2 & 2
\end{bmatrix}
\rref
\begin{bmatrix}
 \leading{1} & 0 & 2 \\
 0 & \leading{1} & -2 \\
 0 & 0 & 0 \\
 0 & 0 & 0
\end{bmatrix}
\end{equation*}
%
From the existence of non-trivial solutions to this system, we can infer non-zero polynomials in $\krn{T}$.  By \acronymref{theorem}{KILT} we then know that $T$ is not injective.\par
%
(b)  Since $3=\dimension{P_2}<\dimension{M_{22}}=4$, by \acronymref{theorem}{SLTD} $T$ is not surjective.\par
%
(c)  Since $T$ is not surjective, it is not invertible by \acronymref{theorem}{ILTIS}.