%%%%(c)
%%%%(c)  This file is a portion of the source for the textbook
%%%%(c)
%%%%(c)    A First Course in Linear Algebra
%%%%(c)    Copyright 2004 by Robert A. Beezer
%%%%(c)
%%%%(c)  See the file COPYING.txt for copying conditions
%%%%(c)
%%%%(c)
This problem asks you to prove that two sets are equal, so use \acronymref{definition}{SE}.\par
%
First show that $\set{\zerovector}\subseteq\eigenspace{A}{\lambda}\cap\eigenspace{A}{\rho}$.  Choose $\vect{x}\in\set{\zerovector}$.  Then $\vect{x}=\zerovector$.   Eigenspaces are subspaces (\acronymref{theorem}{EMS}), so both $\eigenspace{A}{\lambda}$ and $\eigenspace{A}{\rho}$ contain the zero vector, and therefore $\vect{x}\in\eigenspace{A}{\lambda}\cap\eigenspace{A}{\rho}$ (\acronymref{definition}{SI}).\par
%
To show that $\eigenspace{A}{\lambda}\cap\eigenspace{A}{\rho}\subseteq\set{\zerovector}$, suppose that $\vect{x}\in\eigenspace{A}{\lambda}\cap\eigenspace{A}{\rho}$.  Then $\vect{x}$ is an eigenvector of $A$ for both $\lambda$ and $\rho$ (\acronymref{definition}{SI}) and so
%
\begin{align*}
\vect{x}
&=1\vect{x}
&&\text{\acronymref{property}{O}}\\
%
&=\frac{1}{\lambda-\rho}\left(\lambda-\rho\right)\vect{x}
&&\lambda\neq\rho,\ \lambda-\rho\neq 0\\
%
&=\frac{1}{\lambda-\rho}\left(\lambda\vect{x}-\rho\vect{x}\right)
&&\text{\acronymref{property}{DSAC}}\\
%
&=\frac{1}{\lambda-\rho}\left(A\vect{x}-A\vect{x}\right)
&&\text{$\vect{x}$ eigenvector of $A$ for $\lambda$, $\rho$}\\
%
&=\frac{1}{\lambda-\rho}\left(\zerovector\right)\\
%
&=\zerovector
&&\text{\acronymref{theorem}{ZVSM}}\\
%
\end{align*}
%
So $\vect{x}=\zerovector$, and trivially, $\vect{x}\in\set{\zerovector}$.