%%%%(c)
%%%%(c)  This file is a portion of the source for the textbook
%%%%(c)
%%%%(c)    A First Course in Linear Algebra
%%%%(c)    Copyright 2004 by Robert A. Beezer
%%%%(c)
%%%%(c)  See the file COPYING.txt for copying conditions
%%%%(c)
%%%%(c)
If $\lambda=2$ is an eigenvalue of $A$, the matrix $A-2I_4$ will be singular, and its null space will be the eigenspace of $A$.  So we form this matrix and row-reduce,
%
\begin{equation*}
A-2I_4=
\begin{bmatrix}
16 & -15 & 33 & -15\\
-4 & 6 & -6 & 6\\
-9 & 9 & -18 & 9\\
5 & -6 & 9 & -6
\end{bmatrix}
\rref
\begin{bmatrix}
\leading{1} & 0 & 3 & 0\\
0 & \leading{1} & 1 & 1\\
0 & 0 & 0 & 0\\
0 & 0 & 0 & 0
\end{bmatrix}
\end{equation*}
%
With two free variables, we know a basis of the null space (\acronymref{theorem}{BNS}) will contain two vectors.  Thus the null space of $A-2I_4$ has dimension two, and so the eigenspace of $\lambda=2$ has dimension two also (\acronymref{theorem}{EMNS}), $\geomult{A}{2}=2$.
