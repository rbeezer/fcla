%%%%(c)
%%%%(c)  This file is a portion of the source for the textbook
%%%%(c)
%%%%(c)    A First Course in Linear Algebra
%%%%(c)    Copyright 2004 by Robert A. Beezer
%%%%(c)
%%%%(c)  See the file COPYING.txt for copying conditions
%%%%(c)
%%%%(c)
The linear transformation $T$ is surjective if for any $p(x) = \alpha + \beta x + \gamma x^2 + \delta x^3$, there is a vector $\vect{u} = \colvector{a\\b\\c\\d\\e}$ in $\complex{5}$ so that $\lt{T}{\vect{u}} = p(x)$.  We need to be able to solve the system
%
\begin{align*}
a &=\alpha\\
b+c &= \beta\\
c + d &= \gamma\\
d + e &= \delta
\end{align*}
%
This system has an infinite number of solutions, one of which is $a = \alpha$, $b = \beta$, $c = 0$, $d = \gamma$ and $e = \delta - \gamma$, so that 
%
\begin{align*}
\lt{T}{\colvector{\alpha\\ \beta\\0\\ \gamma \\ \delta - \gamma}} 
&= \alpha + (\beta + 0)x + (0 + \gamma) x^2 + (\gamma + (\delta - \gamma)) x^3\\
&= \alpha + \beta x + \gamma x^2 + \delta x^3\\
&= p(x).
\end{align*}
Thus, $T$ is surjective, since for every vector $\vect{v} \in P_3$, there exists a vector $\vect{u} \in \complex{5}$ so that $\lt{T}{\vect{u}} = \vect{v}$.