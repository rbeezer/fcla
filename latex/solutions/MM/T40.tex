%%%%(c)
%%%%(c)  This file is a portion of the source for the textbook
%%%%(c)
%%%%(c)    A First Course in Linear Algebra
%%%%(c)    Copyright 2004 by Robert A. Beezer
%%%%(c)
%%%%(c)  See the file COPYING.txt for copying conditions
%%%%(c)
%%%%(c)
To prove that one set is a subset of another, we start with an element of the smaller set and see if we can determine that it is a member of the larger set (\acronymref{definition}{SSET}).  Suppose $\vect{x}\in\nsp{B}$.  Then we know that $B\vect{x}=\zerovector$ by \acronymref{definition}{NSM}.  Consider
%
\begin{align*}
(AB)\vect{x}&=A(B\vect{x})&&\text{\acronymref{theorem}{MMA}}\\
&=A\zerovector&&\text{Hypothesis}\\
&=\zerovector&&\text{\acronymref{theorem}{MMZM}}\\
\end{align*}
%
This establishes that $\vect{x}\in\nsp{AB}$, so $\nsp{B}\subseteq\nsp{AB}$.\par
%
To show that the inclusion does not hold in the opposite direction, choose $B$ to be any nonsingular matrix of size $n$.  Then $\nsp{B}=\set{\zerovector}$ by \acronymref{theorem}{NMTNS}.  Let $A$ be the square zero matrix, $\zeromatrix$, of the same size.  Then $AB=\zeromatrix B=\zeromatrix$ by \acronymref{theorem}{MMZM} and therefore $\nsp{AB}=\complex{n}$, and is {\em not} a subset of $\nsp{B}=\set{\zerovector}$.
