%%%%(c)
%%%%(c)  This file is a portion of the source for the textbook
%%%%(c)
%%%%(c)    A First Course in Linear Algebra
%%%%(c)    Copyright 2004 by Robert A. Beezer
%%%%(c)
%%%%(c)  See the file COPYING.txt for copying conditions
%%%%(c)
%%%%(c)
Apply \acronymref{theorem}{TSS}.\par
%
First, the zero vector of $M_{nn}$ is the zero matrix, $\zeromatrix$, whose entries are all zero (\acronymref{definition}{ZM}).  This matrix then meets the condition that $\matrixentry{\zeromatrix}{ij}=0$ for $i>j$ and so is an element of $UT_n$.\par
%
Suppose $A,B\in UT_n$.  Is $A+B\in UT_n$?  We examine the entries of $A+B$ ``below'' the diagonal.  That is, in the following, assume that $i>j$.
%
\begin{align*}
\matrixentry{A+B}{ij}
&=\matrixentry{A}{ij}+\matrixentry{B}{ij}&&\text{\acronymref{definition}{MA}}\\
&=0 + 0&&\text{$A,B\in UT_n$}\\
&=0
\end{align*}
%
which qualifies $A+B$ for membership in $UT_n$.\par
%
Suppose $\alpha\in\complex{}$ and $A\in UT_n$.  Is $\alpha A\in UT_n$?  We examine the entries of $\alpha A$ ``below'' the diagonal.  That is, in the following, assume that $i>j$.
%
\begin{align*}
\matrixentry{\alpha A}{ij}
&=\alpha\matrixentry{A}{ij}&&\text{\acronymref{definition}{MSM}}\\
&=\alpha 0&&\text{$A\in UT_n$}\\
&=0
\end{align*}
%
which qualifies $\alpha A$ for membership in $UT_n$.\par
%
Having fulfilled the three conditions of \acronymref{theorem}{TSS} we see that $UT_n$ is a subspace of $M_{nn}$.