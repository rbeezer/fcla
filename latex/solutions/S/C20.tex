%%%%(c)
%%%%(c)  This file is a portion of the source for the textbook
%%%%(c)
%%%%(c)    A First Course in Linear Algebra
%%%%(c)    Copyright 2004 by Robert A. Beezer
%%%%(c)
%%%%(c)  See the file COPYING.txt for copying conditions
%%%%(c)
%%%%(c)
The question is if $p$ can be written as a linear combination of the vectors in $W$.  To check this, we set $p$ equal to a linear combination and massage with the definitions of vector addition and scalar multiplication that we get with $P_3$ (\acronymref{example}{VSP})
%
\begin{align*}
p(x)&=a_1(x^3+x^2+x)+a_2(x^3+2x-6)+a_3(x^2-5)\\
x^3+6x+4&=(a_1+a_2)x^3+(a_1+a_3)x^2+(a_1+2a_2)x+(-6a_2-5a_3)\\
\end{align*}
%
Equating coefficients of equal powers of $x$, we get the system of equations,
%
\begin{align*}
a_1+a_2&=1\\
a_1+a_3&=0\\
a_1+2a_2&=6\\
-6a_2-5a_3&=4
\end{align*}
%
The augmented matrix of this system of equations row-reduces to
%
\begin{equation*}
\begin{bmatrix}
\leading{1} & 0 & 0 & 0\\
0 & \leading{1} & 0 & 0\\
0 & 0 & \leading{1} & 0\\
0 & 0 & 0 & \leading{1}
\end{bmatrix}
\end{equation*}
%
There is a leading 1 in the last column, so \acronymref{theorem}{RCLS} implies that the system is inconsistent.  So there is no way for $p$ to gain membership in $W$, so $p\not\in W$.