%%%%(c)
%%%%(c)  This file is a portion of the source for the textbook
%%%%(c)
%%%%(c)    A First Course in Linear Algebra
%%%%(c)    Copyright 2004 by Robert A. Beezer
%%%%(c)
%%%%(c)  See the file COPYING.txt for copying conditions
%%%%(c)
%%%%(c)
Begin with a matrix $A$ (of any size) that does not have any zero rows, but which when row-reduced to $B$ yields at least one row of zeros.  Such a matrix should be easy to construct (or find, like say from \acronymref{archetype}{A}).\par
%
$\csp{A}$ will contain some vectors whose final slot (entry $m$) is non-zero, however, every column vector from the matrix $B$ will have a zero in slot $m$ and so every vector in $\csp{B}$ will also contain a zero in the final slot.  This means that $\csp{A}\neq\csp{B}$, since we have vectors in $\csp{A}$ that cannot be elements of $\csp{B}$.