%%%%(c)
%%%%(c)  This file is a portion of the source for the textbook
%%%%(c)
%%%%(c)    A First Course in Linear Algebra
%%%%(c)    Copyright 2004 by Robert A. Beezer
%%%%(c)
%%%%(c)  See the file COPYING.txt for copying conditions
%%%%(c)
%%%%(c)
The requested set is described by \acronymref{theorem}{BNS}.  It is easiest to find by using the procedure of \acronymref{example}{VFSAL}.  Begin by row-reducing the matrix, viewing it as the coefficient matrix of a homogeneous system of equations.  We obtain,
%
\begin{equation*}
\begin{bmatrix}
\leading{1} & 0 & 1 & -2\\
0 & \leading{1} & 1 & 1\\
0 & 0 & 0 & 0
\end{bmatrix}
\end{equation*}
%
Now build the vector form of the solutions to this homogeneous system (\acronymref{theorem}{VFSLS}).  The free variables are $x_3$ and $x_4$, corresponding to the columns without leading 1's,
%
\begin{equation*}
\colvector{x_1\\x_2\\x_3\\x_4}=
x_3\colvector{-1\\-1\\1\\0}+
x_4\colvector{2\\-1\\0\\1}
\end{equation*}
%
The desired set $S$ is simply the constant vectors in this expression, and these are the vectors $\vect{z}_1$ and $\vect{z}_2$ described by \acronymref{theorem}{BNS}.
%
\begin{equation*}
S=\set{
\colvector{-1\\-1\\1\\0},\,
\colvector{2\\-1\\0\\1}
}
\end{equation*}
