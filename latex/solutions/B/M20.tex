%%%%(c)
%%%%(c)  This file is a portion of the source for the textbook
%%%%(c)
%%%%(c)    A First Course in Linear Algebra
%%%%(c)    Copyright 2004 by Robert A. Beezer
%%%%(c)
%%%%(c)  See the file COPYING.txt for copying conditions
%%%%(c)
%%%%(c)
We need to establish the linear independence and spanning properties of the set
%
\begin{equation*}
B=\setparts{B_{k\ell}}{1\leq k\leq m,\ 1\leq\ell\leq n}
\end{equation*}
%
relative to the vector space $M_{mn}$.\par
%
This proof is more transparent if you write out individual matrices in the basis with lots of zeros and dots and a lone one.  But we don't have room for that here, so we will use summation notation.  Think carefully about each step, especially when the double summations seem to ``disappear.''  Begin with a relation of linear dependence, using double subscripts on the scalars to align with the basis elements.
%
\begin{equation*}
\zeromatrix=\sum_{k=1}^{m}\sum_{\ell=1}^{n}\alpha_{k\ell}B_{k\ell}
\end{equation*}
%
Now consider the entry in row $i$ and column $j$ for these equal matrices,
%
\begin{align*}
0
&=\matrixentry{\zeromatrix}{ij}&&\text{\acronymref{definition}{ZM}}\\
%
&=\matrixentry{\sum_{k=1}^{m}\sum_{\ell=1}^{n}\alpha_{k\ell}B_{k\ell}}{ij}
&&\text{\acronymref{definition}{ME}}\\
%
&=\sum_{k=1}^{m}\sum_{\ell=1}^{n}\matrixentry{\alpha_{k\ell}B_{k\ell}}{ij}
&&\text{\acronymref{definition}{MA}}\\
%
&=\sum_{k=1}^{m}\sum_{\ell=1}^{n}\alpha_{k\ell}\matrixentry{B_{k\ell}}{ij}
&&\text{\acronymref{definition}{MSM}}\\
%
&=\alpha_{ij}\matrixentry{B_{ij}}{ij}&&
\text{$\matrixentry{B_{k\ell}}{ij}=0$ when $(k,\ell)\neq(i,j)$}\\
%
&=\alpha_{ij}(1)&&\text{$\matrixentry{B_{ij}}{ij}=1$}\\
%
&=\alpha_{ij}
\end{align*}
%
Since $i$ and $j$ were arbitrary, we find that each scalar is zero and so $B$ is linearly independent (\acronymref{definition}{LI}).\par
%
To establish the spanning property of $B$ we need only show that an arbitrary matrix $A$ can be written as a linear combination of the elements of $B$.  So suppose that $A$ is an arbitrary $m\times n$ matrix and consider the matrix $C$ defined as a linear combination of the elements of $B$ by 
%
\begin{equation*}
C=\sum_{k=1}^{m}\sum_{\ell=1}^{n}\matrixentry{A}{k\ell}B_{k\ell}
\end{equation*}
%
Then,
%
\begin{align*}
\matrixentry{C}{ij}
&=\matrixentry{\sum_{k=1}^{m}\sum_{\ell=1}^{n}\matrixentry{A}{k\ell}B_{k\ell}}{ij}
&&\text{\acronymref{definition}{ME}}\\
%
&=\sum_{k=1}^{m}\sum_{\ell=1}^{n}\matrixentry{\matrixentry{A}{k\ell}B_{k\ell}}{ij}
&&\text{\acronymref{definition}{MA}}\\
%
&=\sum_{k=1}^{m}\sum_{\ell=1}^{n}\matrixentry{A}{k\ell}\matrixentry{B_{k\ell}}{ij}
&&\text{\acronymref{definition}{MSM}}\\
%
&=\matrixentry{A}{ij}\matrixentry{B_{ij}}{ij}
&&\text{$\matrixentry{B_{k\ell}}{ij}=0$ when $(k,\ell)\neq(i,j)$}\\
%
&=\matrixentry{A}{ij}(1)&&\text{$\matrixentry{B_{ij}}{ij}=1$}\\
%
&=\matrixentry{A}{ij}
\end{align*}
%
So by \acronymref{definition}{ME}, $A=C$, and therefore $A\in\spn{B}$.  By \acronymref{definition}{B}, the set $B$ is a basis of the vector space $M_{mn}$.
