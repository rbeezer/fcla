%%%%(c)
%%%%(c)  This file is a portion of the source for the textbook
%%%%(c)
%%%%(c)    A First Course in Linear Algebra
%%%%(c)    Copyright 2004 by Robert A. Beezer
%%%%(c)
%%%%(c)  See the file COPYING.txt for copying conditions
%%%%(c)
%%%%(c)
By \acronymref{theorem}{DCM} we know that $\complex{n}$ has dimension $n$.  So by \acronymref{theorem}{G} we need only establish that the set $C$ is linearly independent or a spanning set.  However, the hypotheses also require that $C$ be of size $n$.  We assumed that $B=\set{\vectorlist{x}{n}}$ had size $n$, but there is no guarantee that $C=\set{A\vect{x}_1,\,A\vect{x}_2,\,A\vect{x}_3,\,\dots,\,A\vect{x}_n}$ will have size $n$.  There could be some ``collapsing'' or ``collisions.''\par
%
Suppose we establish that $C$ is linearly independent.  Then $C$ must have $n$ distinct elements or else we could fashion a nontrivial relation of linear dependence involving duplicate elements.\par
%
If we instead to choose to prove that $C$ is a spanning set, then we could establish the uniqueness of the elements of $C$ quite easily.  Suppose that $A\vect{x}_i=A\vect{x}_j$.  Then
%
\begin{align*}
A(\vect{x}_i-\vect{x}_j)&=A\vect{x}_i - A\vect{x}_j=\zerovector
\end{align*}
%
Since $A$ is nonsingular, we conclude that $\vect{x}_i-\vect{x}_j=\zerovector$, or $\vect{x}_i=\vect{x}_j$, contrary to our description of $B$.