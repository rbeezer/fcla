%%%%(c)
%%%%(c)  This file is a portion of the source for the textbook
%%%%(c)
%%%%(c)    A First Course in Linear Algebra
%%%%(c)    Copyright 2004 by Robert A. Beezer
%%%%(c)
%%%%(c)  See the file COPYING.txt for copying conditions
%%%%(c)
%%%%(c)
Represent each of the two systems by a vector equality, $A\vect{x}=\vect{c}$ and $A\vect{y}=\vect{d}$.  Then in the spirit of \acronymref{example}{SABMI}, solutions are given by
%
\begin{align*}
\vect{x}&=B\vect{c}=\colvector{8\\21\\-5\\-16}&
\vect{y}&=B\vect{d}=\colvector{5\\10\\0\\-7}
\end{align*}
%
Notice how we could solve many more systems having $A$ as the coefficient matrix, and how each such system has a unique solution.  You might check your work by substituting the solutions back into the systems of equations, or forming the linear combinations of the columns of $A$ suggested by \acronymref{theorem}{SLSLC}.
