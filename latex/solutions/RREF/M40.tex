%%%%(c)
%%%%(c)  This file is a portion of the source for the textbook
%%%%(c)
%%%%(c)    A First Course in Linear Algebra
%%%%(c)    Copyright 2004 by Robert A. Beezer
%%%%(c)
%%%%(c)  See the file COPYING.txt for copying conditions
%%%%(c)
%%%%(c)
(a)\quad Let $R$ be the common reduced row-echelon form of $B$ and $C$.  A sequence of row operations converts $B$ to $R$ and a second sequence of row operations converts $C$ to $R$. If we ``reverse'' the second sequence's order, and reverse each individual row operation (see \acronymref{exercise}{RREF.T10}) then we can begin with $B$, convert to $R$ with the first sequence, and then convert to $C$ with the reversed sequence.  Satisfying \acronymref{definition}{REM} we can say $B$ and $C$ are row-equivalent matrices.\par
%
(b)\quad We will work this carefully for the first row of $B$ and just give the solution for the next two rows.  For row 1 of $B$ take $i=1$ and we have
%
\begin{align*}
\matrixentry{B}{1j}
&=
\delta_{11}\matrixentry{C}{1j}+
\delta_{12}\matrixentry{C}{2j}+
\delta_{13}\matrixentry{C}{3j}
&
1&\leq j\leq 4
\end{align*}
%
If we substitute the four values for $j$ we arrive at four linear equations in the three unknowns $\delta_{11}, \delta_{12}, \delta_{13}$,
%
\begin{align*}
(j=1)&
&
\matrixentry{B}{11}
&=
\delta_{11}\matrixentry{C}{11}+
\delta_{12}\matrixentry{C}{21}+
\delta_{13}\matrixentry{C}{31}
&
&\Rightarrow
&
1
&=
\delta_{11}(1)+
\delta_{12}(1)+
\delta_{13}(-1)\\
%%
%%
(j=2)&
&
\matrixentry{B}{12}
&=
\delta_{11}\matrixentry{C}{12}+
\delta_{12}\matrixentry{C}{22}+
\delta_{13}\matrixentry{C}{32}
&
&\Rightarrow
&
3
&=
\delta_{11}(2)+
\delta_{12}(1)+
\delta_{13}(-1)\\
%%
%%
(j=3)&
&
\matrixentry{B}{13}
&=
\delta_{11}\matrixentry{C}{13}+
\delta_{12}\matrixentry{C}{23}+
\delta_{13}\matrixentry{C}{33}
&
&\Rightarrow
&
-2
&=
\delta_{11}(1)+
\delta_{12}(4)+
\delta_{13}(-4)\\
%%
%%
(j=4)&
&
\matrixentry{B}{14}
&=
\delta_{11}\matrixentry{C}{14}+
\delta_{12}\matrixentry{C}{24}+
\delta_{13}\matrixentry{C}{34}
&
&\Rightarrow
&
2
&=
\delta_{11}(2)+
\delta_{12}(0)+
\delta_{13}(1)
%
\end{align*}
%
We form the augmented matrix of this system and row-reduce to find the solutions,
%
\begin{align*}
\begin{bmatrix}
1 & 1 & -1 & 1 \\
2 & 1 & -1 & 3 \\
1 & 4 & -4 & -2 \\
2 & 0 & 1 & 2
\end{bmatrix}
&\rref
\begin{bmatrix}
\leading{1} & 0 & 0 & 2 \\
0 & \leading{1} & 0 & -3 \\
0 & 0 & \leading{1} & -2 \\
0 & 0 & 0 & 0
\end{bmatrix}
\end{align*}
%
So the unique solution is $\delta_{11}=2$, $\delta_{12}=-3$, $\delta_{13}=-2$.  Entirely similar work will lead you to
%
\begin{align*}
\delta_{21}&=-1
&
\delta_{22}&=1
&
\delta_{23}&=1
%
\intertext{and}
%
\delta_{31}&=-4
&
\delta_{32}&=8
&
\delta_{33}&=5
%
\end{align*}
%

