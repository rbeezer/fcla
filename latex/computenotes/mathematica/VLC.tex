%%%%(c)
%%%%(c)  This file is a portion of the source for the textbook
%%%%(c)
%%%%(c)    A First Course in Linear Algebra
%%%%(c)    Copyright 2004 by Robert A. Beezer
%%%%(c)
%%%%(c)  See the file COPYING.txt for copying conditions
%%%%(c)
%%%%(c)
\contributedby{\robertbeezer}\\
Vectors in {\sl Mathematica} are represented as lists, written and displayed horizontally.  For example, the vector
%
\begin{equation*}
\vect{v}=\colvector{1\\2\\3\\4}
\end{equation*}
%
would be entered and named via the command
%
\begin{equation*}
\computerfont{v=\{1,\,2,\,3,\,4\}}
\end{equation*}
%
Vector addition and scalar multiplication are then very natural.  If \computerfont{u} and \computerfont{v} are two lists of equal length, then
%
\begin{equation*}
\computerfont{2 u + (-3) v}
\end{equation*}
%
will compute the correct vector and return it as a list.  If \computerfont{u} and \computerfont{v} have different sizes, then {\sl Mathematica} will complain about ``objects of unequal length.''
