%%%%(c)
%%%%(c)  This file is a portion of the source for the textbook
%%%%(c)
%%%%(c)    A First Course in Linear Algebra
%%%%(c)    Copyright 2004 by Robert A. Beezer
%%%%(c)
%%%%(c)  See the file COPYING.txt for copying conditions
%%%%(c)
%%%%(c)
If $A$ and $B$ are matrices defined in {\sl Mathematica}, then \computerfont{A.B} will return the product of the two matrices (notice the dot between the matrices).  If $A$ is a matrix and $\vect{v}$ is a vector, then \computerfont{A.v} will return the vector that is the matrix-vector product of $A$ and $v$.  In every case the sizes of the matrices and vectors need to be correct.\par
%
Some examples:
%
%  align environment and & and font change don't play nice
%
\begin{gather*}
\computerfont{\{\{1,\,2\},\, \{3,\,4\}\}.\{\{5,\,6,\,7\},\,\{8,\,9,\,10\}\} =
\{\{21,\,24,\,27\},\,\{47,\,54,\,61\}\}}\\
\computerfont{\{\{1,\,2\},\, \{3,\,4\}\}.\{\{5\},\,\{6\}\} = \{\{17\},\,\{39\}\}}\\
\computerfont{\{\{1,\,2\},\, \{3,\,4\}\}.\{5,\,6\} = \{17,\,39\}}
\end{gather*}
%
Understanding the difference between the last two examples will go a long way to explaining how some {\sl Mathematica} constructs work.
