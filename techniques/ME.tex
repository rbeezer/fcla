%%%%(c)
%%%%(c)  This file is a portion of the source for the textbook
%%%%(c)
%%%%(c)    A First Course in Linear Algebra
%%%%(c)    Copyright 2004 by Robert A. Beezer
%%%%(c)
%%%%(c)  See the file COPYING.txt for copying conditions
%%%%(c)
%%%%(c)
A very specialized form of a theorem begins with the statement ``The following are equivalent\dots,'' which is then followed by a list of statements.  Informally, this lead-in sometimes gets abbreviated by ``TFAE.''  This formulation means that any two of the statements on the list can be connected with an ``if and only if'' to form a theorem.  So if the list has $n$ statements then, there are $\tfrac{n(n-1)}{2}$ possible equivalences that can be constructed (and are claimed to be true).\par
%
Suppose a theorem of this form has statements denoted as $A$, $B$, $C$,\dots $Z$.  To prove the entire theorem, we can prove $A\Rightarrow B$, $B\Rightarrow C$, $C\Rightarrow D$,\dots, $Y\Rightarrow Z$ and finally, $Z\Rightarrow A$.  This circular chain of $n$ equivalences would allow us, logically, if not practically, to form any one of the $\tfrac{n(n-1)}{2}$ possible equivalences by chasing the equivalences around the circle as far as required.  
