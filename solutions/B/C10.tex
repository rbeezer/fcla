%%%%(c)
%%%%(c)  This file is a portion of the source for the textbook
%%%%(c)
%%%%(c)    A First Course in Linear Algebra
%%%%(c)    Copyright 2004 by Robert A. Beezer
%%%%(c)
%%%%(c)  See the file COPYING.txt for copying conditions
%%%%(c)
%%%%(c)
\acronymref{theorem}{BS} says that if we take these 5 vectors, put them into a matrix, and row-reduce to discover the pivot columns, then the corresponding vectors in $S$ will be linearly independent and span $S$, and thus will form a basis of $S$.
%
\begin{align*}
\begin{bmatrix} 
1 & 1 & 1 & 1 & 3\\
3 & 2 & 1 & 2 & 4\\
2 & 1 & 0 & 2 & 1\\
1 & 1 & 1 & 1 & 3 
\end{bmatrix}
&\rref
\begin{bmatrix}
\leading{1} & 0 & -1 & 0 & -2\\
0 & \leading{1} & 2 & 0 & 5\\
0 & 0 & 0 & \leading{1} & 0\\
0 & 0 & 0 & 0 &0
\end{bmatrix}
\end{align*}
%
Thus, the independent vectors that span $S$ are the first, second and fourth of the set, so a basis of $S$ is 
%
\begin{align*}
B &= \set{
\colvector{1\\3\\2\\1},
\colvector{1\\2\\1\\1},
\colvector{1\\2\\2\\1}
}
\end{align*}
%


