%%%%(c)
%%%%(c)  This file is a portion of the source for the textbook
%%%%(c)
%%%%(c)    A First Course in Linear Algebra
%%%%(c)    Copyright 2004 by Robert A. Beezer
%%%%(c)
%%%%(c)  See the file COPYING.txt for copying conditions
%%%%(c)
%%%%(c)
The conclusion of \acronymref{theorem}{BNS} gives us everything this question asks for.  We need the reduced row-echelon form of the matrix so we can determine the number of vectors in $T$, and their entries.
\begin{align*}
\begin{bmatrix}
 2 & 1 & 1 & 1 \\
 -4 & -3 & 1 & -7 \\
 1 & 1 & -1 & 3
\end{bmatrix}
&\rref
\begin{bmatrix}
 \leading{1} & 0 & 2 & -2 \\
 0 & \leading{1} & -3 & 5 \\
 0 & 0 & 0 & 0
\end{bmatrix}
\end{align*}
%
We can build the set $T$ in immediately via \acronymref{theorem}{BNS}, but we will illustrate its construction in two steps.  Since $F=\set{3,\,4}$, we will have two vectors and can distribute strategically placed ones, and many zeros.  Then we distribute the negatives of the appropriate entries of the non-pivot columns of the reduced row-echelon matrix. 
%
\begin{align*}
T&=\set{
\colvector{\\\\1\\0},\,
\colvector{\\\\0\\1}
}
&
T&=\set{
\colvector{-2\\3\\1\\0},\,
\colvector{2\\-5\\0\\1}
}
\end{align*}