%%%%(c)
%%%%(c)  This file is a portion of the source for the textbook
%%%%(c)
%%%%(c)    A First Course in Linear Algebra
%%%%(c)    Copyright 2004 by Robert A. Beezer
%%%%(c)
%%%%(c)  See the file COPYING.txt for copying conditions
%%%%(c)
%%%%(c)
By \acronymref{definition}{LICV}, we can complete this problem by finding scalars, $\alpha_1,\,\alpha_2,\,\alpha_3$, not all zero, such that
%
\begin{align*}
\alpha_1\left(2\vect{v}_1+3\vect{v}_2+\vect{v}_3\right)+
\alpha_2\left(\vect{v}_1-\vect{v}_2-2\vect{v}_3\right)+
\alpha_3\left(2\vect{v}_1+\vect{v}_2-\vect{v}_3\right)
&=
\zerovector
\end{align*}
%
Using various properties in \acronymref{theorem}{VSPCV}, we can rearrange this vector equation to
%
\begin{align*}
\left(2\alpha_1+\alpha_2+2\alpha_3\right)\vect{v}_1+
\left(3\alpha_1-\alpha_2+\alpha_3\right)\vect{v}_2+
\left(\alpha_1-2\alpha_2-\alpha_3\right)\vect{v}_3
&=
\zerovector
%
\end{align*}
%
We can certainly make this vector equation true if we can determine values for the $\alpha$'s such that
%
\begin{align*}
2\alpha_1+\alpha_2+2\alpha_3&=0\\
3\alpha_1-\alpha_2+\alpha_3&=0\\
\alpha_1-2\alpha_2-\alpha_3&=0
\end{align*}
%
Aah, a homogeneous system of equations.  And it has infinitely many non-zero solutions.  By the now familiar techniques, one such solution is $\alpha_1=3$, $\alpha_2=4$, $\alpha_3=-5$, which you can check in the original relation of linear dependence on $T$ above.\par
%
Note that simply writing down the three scalars, and demonstrating that they provide a nontrivial relation of linear dependence on $T$, could be considered an ironclad solution.  But it wouldn't have been very informative for you if we had only done just that here.  Compare this solution very carefully with \acronymref{solution}{LI.M21}.