%%%%(c)
%%%%(c)  This file is a portion of the source for the textbook
%%%%(c)
%%%%(c)    A First Course in Linear Algebra
%%%%(c)    Copyright 2004 by Robert A. Beezer
%%%%(c)
%%%%(c)  See the file COPYING.txt for copying conditions
%%%%(c)
%%%%(c)
\acronymref{theorem}{BNS} provides formulas for $n-r$ vectors that will meet the requirements of this question.  These vectors are the same ones listed in \acronymref{theorem}{VFSLS} when we solve the homogeneous system $\homosystem{A}$, whose solution set is the null space (\acronymref{definition}{NSM}).\par
%
To apply \acronymref{theorem}{BNS} or \acronymref{theorem}{VFSLS} we first row-reduce the matrix, resulting in 
%
\begin{equation*}
B=
\begin{bmatrix}
 \leading{1} & 2 & 0 & 0 & 3 \\
 0 & 0 & \leading{1} & 0 & 6 \\
 0 & 0 & 0 & \leading{1} & -4 \\
 0 & 0 & 0 & 0 & 0
\end{bmatrix}
\end{equation*}
%
So we see that $n-r=5-3=2$ and $F=\set{2,5}$, so the vector form of a generic solution vector is
%
\begin{equation*}
\colvector{x_1\\x_2\\x_3\\x_4\\x_5}
=
x_2\colvector{-2\\1\\ 0\\ 0\\0}
+
x_5\colvector{-3\\0\\-6 \\4 \\1}
\end{equation*}
%
So we have
%
\begin{equation*}
\nsp{A}=\spn{\set{
\colvector{-2\\1\\ 0\\ 0\\0},\,
\colvector{-3\\0\\-6 \\4 \\1}
}}
\end{equation*}
%
