%%%%(c)
%%%%(c)  This file is a portion of the source for the textbook
%%%%(c)
%%%%(c)    A First Course in Linear Algebra
%%%%(c)    Copyright 2004 by Robert A. Beezer
%%%%(c)
%%%%(c)  See the file COPYING.txt for copying conditions
%%%%(c)
%%%%(c)
If $x$, $y$ and $z$ represent the money held by Dan, Diane and Donna, then $y=15-z$ and $x=20-y=20-(15-z)=5+z$.  We can let $z$ take on any value from $0$ to $15$ without any of the three amounts being negative, since presumably middle-schoolers are too young to assume debt.\par
%
Then the total capital held by the three is $x+y+z=(5+z)+(15-z)+z=20+z$.  So their combined holdings can range anywhere from \$20 (Donna is broke) to \$35 (Donna is flush).\par
%
We will have more to say about this situation in \acronymref{section}{TSS}, and specifically \acronymref{theorem}{CMVEI}.