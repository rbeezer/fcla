%%%%(c)
%%%%(c)  This file is a portion of the source for the textbook
%%%%(c)
%%%%(c)    A First Course in Linear Algebra
%%%%(c)    Copyright 2004 by Robert A. Beezer
%%%%(c)
%%%%(c)  See the file COPYING.txt for copying conditions
%%%%(c)
%%%%(c)
Since $\linearsystem{A}{b}$ has at least one solution, we can apply \acronymref{theorem}{PSPHS}.  Because the solution is assumed to be unique, the null space of $A$ must be trivial.  Then \acronymref{theorem}{NMTNS} implies that $A$ is nonsingular.\par
%
The converse of this statement is a trivial application of \acronymref{theorem}{NMUS}.  That said, we could extend our NSMxx series of theorems with an added equivalence for nonsingularity, ``Given a single vector of constants, $\vect{b}$, the system $\linearsystem{A}{\vect{b}}$ has a unique solution.''