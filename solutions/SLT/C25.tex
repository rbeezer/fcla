%%%%(c)
%%%%(c)  This file is a portion of the source for the textbook
%%%%(c)
%%%%(c)    A First Course in Linear Algebra
%%%%(c)    Copyright 2004 by Robert A. Beezer
%%%%(c)
%%%%(c)  See the file COPYING.txt for copying conditions
%%%%(c)
%%%%(c)
To find the range of $T$, apply $T$ to the elements of a spanning set for $\complex{3}$ as suggested in \acronymref{theorem}{SSRLT}.  We will use the standard basis vectors (\acronymref{theorem}{SUVB}).
%
\begin{equation*}
\rng{T}=
\spn{\set{\lt{T}{\vect{e}_1},\,\lt{T}{\vect{e}_2},\,\lt{T}{\vect{e}_3}}}=
\spn{\set{\colvector{2\\-4},\,\colvector{-1\\2},\,\colvector{5\\-10}}}
\end{equation*}
%
Each of these vectors is a scalar multiple of the others, so we can toss two of them in reducing the spanning set to a linearly independent set (or be more careful and apply \acronymref{theorem}{BCS} on a matrix with these three vectors as columns).  The result is the basis of the range,
%
\begin{equation*}
\set{\colvector{1\\-2}}
\end{equation*}
\par
%
With $\rank{T}\neq 2$, $\rng{T}\neq\complex{2}$, so \acronymref{theorem}{RSLT} says $T$ is not surjective.
