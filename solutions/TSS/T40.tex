%%%%(c)
%%%%(c)  This file is a portion of the source for the textbook
%%%%(c)
%%%%(c)    A First Course in Linear Algebra
%%%%(c)    Copyright 2004 by Robert A. Beezer
%%%%(c)
%%%%(c)  See the file COPYING.txt for copying conditions
%%%%(c)
%%%%(c)
Since the system is consistent, we know there is either a unique solution, or infinitely many solutions (\acronymref{theorem}{PSSLS}).  If we perform row operations (\acronymref{definition}{RO}) on the augmented matrix of the system, the two equal columns of the coefficient matrix will suffer the same fate, and remain equal in the final reduced row-echelon form.  Suppose both of these columns are pivot columns (\acronymref{definition}{RREF}).  Then there is single row containing the two leading 1's of the two pivot columns, a violation of reduced row-echelon form (\acronymref{definition}{RREF}).  So at least one of these columns is not a pivot column, and the column index indicates a free variable in the description of the solution set (\acronymref{definition}{IDV}).  With a free variable, we arrive at an infinite solution set (\acronymref{theorem}{FVCS}).