%%%%(c)
%%%%(c)  This file is a portion of the source for the textbook
%%%%(c)
%%%%(c)    A First Course in Linear Algebra
%%%%(c)    Copyright 2004 by Robert A. Beezer
%%%%(c)
%%%%(c)  See the file COPYING.txt for copying conditions
%%%%(c)
%%%%(c)
Check the two conditions of \acronymref{definition}{LT}.
%
\begin{align*}
\lt{T}{\vect{u}+\vect{v}}
&=\lt{T}{\left(a+bx+cx^2\right)+\left(d+ex+fx^2\right)}\\
&=\lt{T}{\left(a+d\right)+\left(b+e\right)x+\left(c+f\right)x^2}\\
&=\colvector{2(a+d)-(b+e)\\(b+e)+(c+f)}\\
&=\colvector{(2a-b)+(2d-e)\\(b+c)+(e+f)}\\
&=\colvector{2a-b\\b+c}+\colvector{2d-e\\e+f}\\
&=\lt{T}{\vect{u}}+\lt{T}{\vect{v}}
%
\intertext{and}
%
\lt{T}{\alpha\vect{u}}
&=\lt{T}{\alpha\left(a+bx+cx^2\right)}\\
&=\lt{T}{\left(\alpha a\right)+\left(\alpha b\right)x+\left(\alpha c\right)x^2}\\
&=\colvector{2(\alpha a)-(\alpha b)\\(\alpha b)+(\alpha c)}\\
&=\colvector{\alpha(2a-b)\\\alpha(b+c)}\\
&=\alpha\colvector{2a-b\\b+c}\\
&=\alpha\lt{T}{\vect{u}}
\end{align*}
%
So $T$ is indeed a linear transformation.
