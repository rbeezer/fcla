%%%%(c)
%%%%(c)  This file is a portion of the source for the textbook
%%%%(c)
%%%%(c)    A First Course in Linear Algebra
%%%%(c)    Copyright 2004 by Robert A. Beezer
%%%%(c)
%%%%(c)  See the file COPYING.txt for copying conditions
%%%%(c)
%%%%(c)
Using a calculator, we find that $A$ has three distinct eigenvalues, $\lambda=3,\,2,\,-1$, with $\lambda=2$ having algebraic multiplicity two, $\algmult{A}{2}=2$.  The eigenvalues $\lambda=3,\,-1$ have algebraic multiplicity one, and so by \acronymref{theorem}{ME} we can conclude that their geometric multiplicities are one as well.  Together with the computation of the geometric multiplicity of $\lambda=2$ from \acronymref{exercise}{EE.C20}, we know
%
\begin{align*}
\geomult{A}{3}&=\algmult{A}{3}=1&
\geomult{A}{2}&=\algmult{A}{2}=2&
\geomult{A}{-1}&=\algmult{A}{-1}=1
\end{align*}
%
This satisfies the hypotheses of \acronymref{theorem}{DMFE}, and so we can conclude that $A$ is diagonalizable.\par
%
A calculator will give us four eigenvectors of $A$, the two for $\lambda=2$ being linearly independent presumably.  Or, by hand, we could find basis vectors for the three eigenspaces.  For $\lambda=3,\,-1$ the eigenspaces have dimension one, and so any eigenvector for these eigenvalues will be multiples of the ones we use below.  For $\lambda=2$ there are many different bases for the eigenspace, so your answer could vary.  Our eigenvectors are the basis vectors we would have obtained if we had actually constructed a basis in \acronymref{exercise}{EE.C20} rather than just computing the dimension.\par
%
By the construction in the proof of \acronymref{theorem}{DC}, the required matrix $S$ has columns that are four linearly independent eigenvectors of $A$ and the diagonal matrix has the eigenvalues on the diagonal (in the same order as the eigenvectors in $S$).  Here are the pieces, ``doing'' the diagonalization,
%
\begin{equation*}
\inverse{
\begin{bmatrix}
-1 & 0 & -3 & 6\\ 
-2 & -1 & -1 & 0\\ 
0 & 0 & 1 & -3\\ 
1 & 1 & 0 & 1
\end{bmatrix}
}
\begin{bmatrix}
18 & -15 & 33 & -15\\
-4 & 8 & -6 & 6\\
-9 & 9 & -16 & 9\\
5 & -6 & 9 & -4
\end{bmatrix}
\begin{bmatrix}
-1 & 0 & -3 & 6\\ 
-2 & -1 & -1 & 0\\ 
0 & 0 & 1 & -3\\ 
1 & 1 & 0 & 1
\end{bmatrix}
=
\begin{bmatrix}
3 & 0 & 0 & 0\\ 
0 & 2 & 0 & 0\\ 
0 & 0 & 2 & 0\\ 
0 & 0 & 0 & -1
\end{bmatrix}
\end{equation*}
%
