%%%%(c)
%%%%(c)  This file is a portion of the source for the textbook
%%%%(c)
%%%%(c)    A First Course in Linear Algebra
%%%%(c)    Copyright 2004 by Robert A. Beezer
%%%%(c)
%%%%(c)  See the file COPYING.txt for copying conditions
%%%%(c)
%%%%(c)
Note in the following that the scalar multiple of a matrix is equivalent to multiplying each of the rows by that scalar, so we actually apply \acronymref{theorem}{DRCM} multiple times below (and are passing up an opportunity to do a proof by induction in the process, which maybe you'd like to do yourself?).
%
\begin{align*}
\charpoly{A}{x}
&=\detname{A-xI_n}&&\text{\acronymref{definition}{CP}}\\
&=\detname{(-1)(xI_n-A)}&&\text{\acronymref{definition}{MSM}}\\
&=(-1)^{n}\detname{xI_n-A}&&\text{\acronymref{theorem}{DRCM}}\\
&=(-1)^{n}r_A(x)
\end{align*}
%
Since the polynomials are scalar multiples of each other, their roots will be identical, so either polynomial could be used in \acronymref{theorem}{EMRCP}.\par
%
Computing by hand, our definition of the characteristic polynomial is easier to use, as you only need to subtract $x$ down the diagonal of the matrix before computing the determinant.  However, the price to be paid is that for odd values of $n$, the coefficient of $x^{n}$ is $-1$, while $r_A(x)$ always has the coefficient $1$ for $x^{n}$ (we say $r_A(x)$ is ``monic.'')