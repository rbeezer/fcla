%%%%(c)
%%%%(c)  This file is a portion of the source for the textbook
%%%%(c)
%%%%(c)    A First Course in Linear Algebra
%%%%(c)    Copyright 2004 by Robert A. Beezer
%%%%(c)
%%%%(c)  See the file COPYING.txt for copying conditions
%%%%(c)
%%%%(c)
Let $\vect{v}$ be an eigenvector of $T$ for the eigenvalue $\lambda$.  Then,
%
\begin{align*}
\lt{\ltinverse{T}}{\vect{v}}&=
\frac{1}{\lambda}\lambda\lt{\ltinverse{T}}{\vect{v}}&&\text{$\lambda\neq 0$}\\
&=\frac{1}{\lambda}\lt{\ltinverse{T}}{\lambda\vect{v}}&&\text{\acronymref{theorem}{ILTLT}}\\
&=\frac{1}{\lambda}\lt{\ltinverse{T}}{\lt{T}{\vect{v}}}&&\text{$\vect{v}$ eigenvector of $T$}\\
&=\frac{1}{\lambda}\lt{I_V}{\vect{v}}&&\text{\acronymref{definition}{IVLT}}\\
&=\frac{1}{\lambda}\vect{v}&&\text{\acronymref{definition}{IDLT}}
\end{align*}
%
which says that $\displaystyle\frac{1}{\lambda}$ is an eigenvalue of $\ltinverse{T}$ with eigenvector $\vect{v}$.  Note that it is possible to prove that any eigenvalue of an invertible linear transformation is never zero.  So the hypothesis that $\lambda$ be nonzero is just a convenience for this problem.
