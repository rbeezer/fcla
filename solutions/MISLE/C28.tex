%%%%(c)
%%%%(c)  This file is a portion of the source for the textbook
%%%%(c)
%%%%(c)    A First Course in Linear Algebra
%%%%(c)    Copyright 2004 by Robert A. Beezer
%%%%(c)
%%%%(c)  See the file COPYING.txt for copying conditions
%%%%(c)
%%%%(c)
Employ \acronymref{theorem}{CINM},
%
\begin{equation*}
\begin{bmatrix}
 1 & 1 & 3 & 1     &   1 & 0 & 0 & 0\\
 -2 & -1 & -4 & -1 &  0 & 1 & 0 & 0\\
 1 & 4 & 10 & 2   &   0 & 0 & 1 & 0\\
 -2 & 0 & -4 & 5	&	 0 & 0 & 0 & 1
 \end{bmatrix}
 \rref
 \begin{bmatrix}
 \leading{1} & 0 & 0 & 0	&	38 & 18 & -5 & -2\\
 0 & \leading{1} & 0 & 0 &	96 & 47 & -12 & -5\\
 0 & 0 & \leading{1} & 0	&	-39 & -19 & 5 & 2\\
 0 & 0 & 0 & \leading{1}	&	-16 & -8 & 2 & 1
 \end{bmatrix}
\end{equation*}
%
And therefore we see that $C$ is nonsingular ($C$ row-reduces to the identity matrix, \acronymref{theorem}{NMRRI}) and by \acronymref{theorem}{CINM},
%
\begin{equation*}
\inverse{C}=
\begin{bmatrix}
38 & 18 & -5 & -2\\
96 & 47 & -12 & -5\\
-39 & -19 & 5 & 2\\
-16 & -8 & 2 & 1
\end{bmatrix}
\end{equation*}