%%%%(c)
%%%%(c)  This file is a portion of the source for the textbook
%%%%(c)
%%%%(c)    A First Course in Linear Algebra
%%%%(c)    Copyright 2004 by Robert A. Beezer
%%%%(c)
%%%%(c)  See the file COPYING.txt for copying conditions
%%%%(c)
%%%%(c)
This is an equality of sets, so \acronymref{definition}{SE} applies.\par
%
The ``easy'' half first.  Show that $X=\spn{\set{\vect{v}_1+\vect{v}_2,\,\vect{v}_1-\vect{v}_2}}\subseteq
\spn{\set{\vect{v}_1,\,\vect{v}_2}}=Y$.\\
%
Choose $\vect{x}\in X$.  Then 
$\vect{x}=a_1(\vect{v}_1+\vect{v}_2)+a_2(\vect{v}_1-\vect{v}_2)$ for some scalars $a_1$ and $a_2$.  Then,
%
\begin{align*}
\vect{x}
&=a_1(\vect{v}_1+\vect{v}_2)+a_2(\vect{v}_1-\vect{v}_2)\\
&=a_1\vect{v}_1+a_1\vect{v}_2+a_2\vect{v}_1+(-a_2)\vect{v}_2\\
&=(a_1+a_2)\vect{v}_1+(a_1-a_2)\vect{v}_2
\end{align*}
%
which qualifies $\vect{x}$ for membership in $Y$, as it is a linear combination of $\vect{v}_1,\,\vect{v}_2$.\par
%
Now show the opposite inclusion, $Y=\spn{\set{\vect{v}_1,\,\vect{v}_2}}\subseteq\spn{\set{\vect{v}_1+\vect{v}_2,\,\vect{v}_1-\vect{v}_2}}=X$.\\
%
Choose $\vect{y}\in Y$.  Then there are scalars $b_1,\,b_2$ such that $ \vect{y}=b_1\vect{v}_1+b_2\vect{v}_2 $.  Rearranging, we obtain,
%
\begin{align*}
\vect{y}
&=b_1\vect{v}_1+b_2\vect{v}_2\\
%
&=\frac{b_1}{2}\left[\left(\vect{v}_1+\vect{v}_2\right)+\left(\vect{v}_1-\vect{v}_2\right)\right]
     +
\frac{b_2}{2}\left[\left(\vect{v}_1+\vect{v}_2\right)-\left(\vect{v}_1-\vect{v}_2\right)\right]\\
%
&=\frac{b_1+b_2}{2}\left(\vect{v}_1+\vect{v}_2\right)+\frac{b_1-b_2}{2}\left(\vect{v}_1-\vect{v}_2\right)
%
\end{align*}
%
This is an expression for $\vect{y}$ as a linear combination of $\vect{v}_1+\vect{v}_2$ and $\vect{v}_1-\vect{v}_2$, earning $\vect{y}$ membership in $X$.
%
Since $X$ is a subset of $Y$, and vice versa, we see that $X=Y$, as desired.