%%%%(c)
%%%%(c)  This file is a portion of the source for the textbook
%%%%(c)
%%%%(c)    A First Course in Linear Algebra
%%%%(c)    Copyright 2004 by Robert A. Beezer
%%%%(c)
%%%%(c)  See the file COPYING.txt for copying conditions
%%%%(c)
%%%%(c)
Let $A$ be the matrix whose columns are the vectors in $T$.  Then row-reduce $A$,
%
\begin{equation*}
A\rref B=
\begin{bmatrix}
 \leading{1} & 0 & 0 & 2 \\
 0 & \leading{1} & 0 & -1 \\
 0 & 0 & \leading{1} & 1
\end{bmatrix}
\end{equation*}
%
From \acronymref{theorem}{BS} we can form $R$ by choosing the columns of $A$ that correspond to the pivot columns of $B$.  \acronymref{theorem}{BS} also guarantees that $R$ will be linearly independent.
%
\begin{equation*}
R=\set{
\colvector{1 \\ -1 \\ 2},\,
\colvector{3 \\ 0 \\ 1},\,
\colvector{4 \\ 2 \\ 3}
}
\end{equation*}
%
That was easy.  To find $S$ will require a bit more work.  From $B$ we can obtain a solution to $\homosystem{A}$, which by \acronymref{theorem}{SLSLC} will provide a nontrivial relation of linear dependence on the columns of $A$, which are the vectors in $T$.  To wit,  choose the free variable $x_4$ to be 1, then $x_1=-2$, $x_2=1$, $x_3=-1$, and so
%
\begin{equation*}
(-2)\colvector{1 \\ -1 \\ 2}+
(1)\colvector{3 \\ 0 \\ 1}+
(-1)\colvector{4 \\ 2 \\ 3}+
(1)\colvector{3 \\ 0 \\ 6}
=
\colvector{0\\0\\0}
\end{equation*}
%
this equation can be rewritten with the second vector staying put, and the other three moving to the other side of the equality,
%
\begin{equation*}
\colvector{3 \\ 0 \\ 1}
=
(2)\colvector{1 \\ -1 \\ 2}+
(1)\colvector{4 \\ 2 \\ 3}+
(-1)\colvector{3 \\ 0 \\ 6}
\end{equation*}
%
We could have chosen other vectors to stay put, but may have then needed to divide by a nonzero scalar.   This equation is enough to conclude that the second vector in $T$ is ``surplus'' and can be replaced (see the careful argument in \acronymref{example}{RSC5}).  So set
%
\begin{equation*}
S=\set{
\colvector{1 \\ -1 \\ 2},\,
\colvector{4 \\ 2 \\ 3},\,
\colvector{3 \\ 0 \\ 6}
}
\end{equation*}
%
and then $\spn{S}=\spn{T}$.  $T$ is also a linearly independent set, which we can show directly.  Make a matrix $C$ whose columns are the vectors in $S$.  Row-reduce $B$ and you will obtain the identity matrix $I_3$.  By \acronymref{theorem}{LIVRN}, the set $S$ is linearly independent.
