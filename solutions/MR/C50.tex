%%%%(c)
%%%%(c)  This file is a portion of the source for the textbook
%%%%(c)
%%%%(c)    A First Course in Linear Algebra
%%%%(c)    Copyright 2004 by Robert A. Beezer
%%%%(c)
%%%%(c)  See the file COPYING.txt for copying conditions
%%%%(c)
%%%%(c)
As usual, build any matrix representation of $L$, most likely using a ``nice'' bases, such as
%
\begin{align*}
B&=\set{
\begin{bmatrix} 1 & 0 \\ 0 & 0 \end{bmatrix},\,
\begin{bmatrix} 0 & 1 \\ 0 & 0 \end{bmatrix},\,
\begin{bmatrix} 0 & 0 \\ 1 & 0 \end{bmatrix},\,
\begin{bmatrix} 0 & 0 \\ 0 & 1 \end{bmatrix}
}\\
C&=\set{1,\,x,\,x^2}
\end{align*}
%
Then the matrix representation (\acronymref{definition}{MR}) is,
%
\begin{equation*}
\matrixrep{L}{B}{C}=
\begin{bmatrix}
 1 & 2 & 4 & 1 \\
 3 & 0 & 1 & -2 \\
 -1 & 1 & 3 & 3
\end{bmatrix}
\end{equation*}
%
\acronymref{theorem}{RCSI} tells us that we can compute the column space of the matrix representation, then use the isomorphism $\vectrepinvname{C}$ to convert the column space of the matrix representation into the range of the linear transformation.  So we first analyze the matrix representation,
%
\begin{equation*}
\begin{bmatrix}
 1 & 2 & 4 & 1 \\
 3 & 0 & 1 & -2 \\
 -1 & 1 & 3 & 3
\end{bmatrix}
\rref
\begin{bmatrix}
 \leading{1} & 0 & 0 & -1 \\
 0 & \leading{1} & 0 & -1 \\
 0 & 0 & \leading{1} & 1
\end{bmatrix}
\end{equation*}
%
With three nonzero rows in the reduced row-echelon form of the matrix, we know the column space has dimension 3.  Since $P_2$ has dimension 3 (\acronymref{theorem}{DP}), the range must be all of $P_2$.  So {\em any} basis of $P_2$ would suffice as a basis for the range.  For instance, $C$ itself would be a correct answer.\par
%
A more laborious approach would be to use \acronymref{theorem}{BCS} and choose the first three columns of the matrix representation as a basis for the range of the matrix representation.  These could then be ``un-coordinatized'' with $\vectrepinvname{C}$ to yield a (``not nice'') basis for $P_2$.