%%%%(c)
%%%%(c)  This file is a portion of the source for the textbook
%%%%(c)
%%%%(c)    A First Course in Linear Algebra
%%%%(c)    Copyright 2004 by Robert A. Beezer
%%%%(c)
%%%%(c)  See the file COPYING.txt for copying conditions
%%%%(c)
%%%%(c)
Choose bases $B$ and $C$ for the matrix representation,
\begin{align*}
B&=\set{1,\,x,\,x^2}
&
C&=\set{
\begin{bmatrix}1 & 0 \\ 0 & 0\end{bmatrix},\,
\begin{bmatrix}0 & 1 \\ 0 & 0\end{bmatrix},\,
\begin{bmatrix}0 & 0 \\ 1 & 0\end{bmatrix},\,
\begin{bmatrix}0 & 0 \\ 0 & 1\end{bmatrix}
}\\
%
\end{align*}
%
Input to $T$ the vectors of the basis $B$ and coordinatize the outputs relative to $C$,
%
\begin{align*}
\vectrep{C}{\lt{T}{1}}&=
\vectrep{C}{\begin{bmatrix}1 & 2 \\ -1 & 3\end{bmatrix}}
=
\vectrep{C}{
1\begin{bmatrix}1 & 0 \\ 0 & 0\end{bmatrix}+
2\begin{bmatrix}0 & 1 \\ 0 & 0\end{bmatrix}+
(-1)\begin{bmatrix}0 & 0 \\ 1 & 0\end{bmatrix}+
3\begin{bmatrix}0 & 0 \\ 0 & 1\end{bmatrix}
}
=
\colvector{1\\2\\-1\\3}\\
%
\vectrep{C}{\lt{T}{x}}&=
\vectrep{C}{\begin{bmatrix}2 & 2 \\ 1 & 2 \end{bmatrix}}
=
\vectrep{C}{
2\begin{bmatrix}1 & 0 \\ 0 & 0\end{bmatrix}+
2\begin{bmatrix}0 & 1 \\ 0 & 0\end{bmatrix}+
1\begin{bmatrix}0 & 0 \\ 1 & 0\end{bmatrix}+
2\begin{bmatrix}0 & 0 \\ 0 & 1\end{bmatrix}
}
=
\colvector{2\\2\\1\\2}\\
%
\vectrep{C}{\lt{T}{x^2}}&=
\vectrep{C}{\begin{bmatrix}-2 & 0 \\ -4 & 2\end{bmatrix}}
=
\vectrep{C}{
(-2)\begin{bmatrix}1 & 0 \\ 0 & 0\end{bmatrix}+
0\begin{bmatrix}0 & 1 \\ 0 & 0\end{bmatrix}+
(-4)\begin{bmatrix}0 & 0 \\ 1 & 0\end{bmatrix}+
2\begin{bmatrix}0 & 0 \\ 0 & 1\end{bmatrix}
}
=
\colvector{-2\\0\\-4\\2}\\
%
\end{align*}
%
Applying \acronymref{definition}{MR} we have the matrix representation
%
\begin{equation*}
\matrixrep{T}{B}{C}=
\begin{bmatrix}
 1 & 2 & -2 \\
 2 & 2 & 0 \\
 -1 & 1 & -4 \\
 3 & 2 & 2
\end{bmatrix}
\end{equation*}
%
The null space of the matrix representation is isomorphic (via $\vectrepname{B}$) to the kernel of the linear transformation (\acronymref{theorem}{KNSI}).  So we compute the null space of the matrix representation by first row-reducing the matrix to,
%
\begin{equation*}
\begin{bmatrix}
 \leading{1} & 0 & 2 \\
 0 & \leading{1} & -2 \\
 0 & 0 & 0 \\
 0 & 0 & 0
\end{bmatrix}
\end{equation*}
%
Employing \acronymref{theorem}{BNS} we have 
%
\begin{equation*}
\nsp{\matrixrep{T}{B}{C}}=\spn{\set{\colvector{-2\\2\\1}}}
\end{equation*}
%
We only need to uncoordinatize this one basis vector to get a basis for $\krn{T}$,
%
\begin{equation*}
\krn{T}
=\spn{\set{\vectrepinv{B}{\colvector{-2\\2\\1}}}}
=\spn{\set{-2+2x+x^2}}
\end{equation*}