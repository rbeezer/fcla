%%%%(c)
%%%%(c)  This file is a portion of the source for the textbook
%%%%(c)
%%%%(c)    A First Course in Linear Algebra
%%%%(c)    Copyright 2004 by Robert A. Beezer
%%%%(c)
%%%%(c)  See the file COPYING.txt for copying conditions
%%%%(c)
%%%%(c)
Choose bases $B$ and $C$ for $M_{12}$ and $M_{21}$ (respectively),
%
\begin{align*}
B=\set{
\begin{bmatrix}1 & 0\end{bmatrix},\,
\begin{bmatrix}0 & 1\end{bmatrix}
}
%
C=\set{
\begin{bmatrix}1 \\ 0\end{bmatrix},\,
\begin{bmatrix}0 \\ 1\end{bmatrix}
}
\end{align*}
%
The resulting matrix representation is 
%
\begin{equation*}
\matrixrep{R}{B}{C}
=
\begin{bmatrix}
1 & 3\\
4 & 11
\end{bmatrix}
\end{equation*}
%
This matrix is invertible (its determinant is nonzero, \acronymref{theorem}{SMZD}), so by \acronymref{theorem}{IMR}, we can compute the matrix representation of $\ltinverse{R}$ with a matrix inverse (\acronymref{theorem}{TTMI}),
%
\begin{equation*}
\matrixrep{\ltinverse{R}}{C}{B}
=\inverse{\begin{bmatrix}1 & 3\\4 & 11\end{bmatrix}}
=\begin{bmatrix}-11 & 3\\4 & -1\end{bmatrix}
\end{equation*}
%
To obtain a general formula for $\ltinverse{R}$, use \acronymref{theorem}{FTMR},
%
\begin{align*}
\lt{\ltinverse{R}}{\begin{bmatrix}x \\ y\end{bmatrix}}
&=\vectrepinv{B}{\matrixrep{\ltinverse{R}}{C}{B}\vectrep{C}{\begin{bmatrix}x\\y\end{bmatrix}}}\\
%
&=\vectrepinv{B}{\begin{bmatrix}-11&3\\4&-1\end{bmatrix}\colvector{x\\y}}\\
%
&=\vectrepinv{B}{\colvector{-11x+3y\\4x-y}}\\
%
&=\begin{bmatrix}-11x+3y&4x-y\end{bmatrix}
%
\end{align*}