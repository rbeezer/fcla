%%%%(c)
%%%%(c)  This file is a portion of the source for the textbook
%%%%(c)
%%%%(c)    A First Course in Linear Algebra
%%%%(c)    Copyright 2004 by Robert A. Beezer
%%%%(c)
%%%%(c)  See the file COPYING.txt for copying conditions
%%%%(c)
%%%%(c)
No matter what the elements of the set $S$ are, we can choose the scalars in a linear combination to all be zero.  Suppose that $S=\set{\vectorlist{v}{p}}$.  Then compute
%
\begin{align*}
0\vect{v}_1+0\vect{v}_2+0\vect{v}_3+\cdots+0\vect{v}_p
&=\zerovector+\zerovector+\zerovector+\cdots+\zerovector\\
&=\zerovector
\end{align*}
%
But what if we choose $S$ to be the empty set?  The {\em convention} is that the empty sum in \acronymref{definition}{SSCV} evaluates to ``zero,'' in this case this is the zero vector.  